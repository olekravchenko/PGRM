\documentclass[a4paper,12pt]{article}
\usepackage[T2A]{fontenc}
\usepackage[utf8]{inputenc}
\usepackage[russian,english]{babel}
\usepackage{amssymb}
\usepackage{amsmath}


\title{Petrov-Galerkin-Rvachev Method.\\C Implementation.}
\author{}
\date{}

\pdfinfo{%
  /Title    ()
  /Author   ()
  /Creator  ()
  /Producer ()
  /Subject  ()
  /Keywords ()
}

\begin{document}
\maketitle
\section{Intro}

\section{Method description}
%description of Petrov-Galerkin Method in general
\subsection{RFM and structures of solution}
%short introduction in R-functions and to idea of structures
\subsection{Dirichlet problem}
%decription of Kantorovich structure
\subsection{Neumann problem}
%development and description of stucture for Neumann problem
\subsection{Mixed boundary problem}
%description of mixed boundary problem structure(possibly with development)


\section{Code description}
\subsection{Structure of code}
Current implementation of PGRM consists of two main parts, which will be described further: Core and Task parts.
\subsubsection{Core part}
Includes the most part of code resporsible for solving Poissons equation. Includes further files: [file list with description of its content]
\subsubsection{Task part}
Consists of "tasks.c" and "tasks.h" only.
"tasks.h" is header file with a description of the most important "public" parts, that can be used for description of a particular boundary problem for Poisson's equation
\subsection{Desfunction of functions}
\subsubsection{Core part}
\subsubsection{Task part}

\section{Build instructions}

\section{Usage example}
\end{document}
